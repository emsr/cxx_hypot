\documentclass[ebook,10pt,oneside,openany,final]{memoir}

\usepackage[american]
           {babel}        % needed for iso dates
\usepackage[iso,american]
           {isodate}      % use iso format for dates
\usepackage[final]
           {listings}     % code listings
\usepackage{longtable}    % auto-breaking tables
\usepackage{ltcaption}    % fix captions for long tables
\usepackage{relsize}      % provide relative font size changes
\usepackage{textcomp}     % provide \text{l,r}angle
\usepackage{underscore}   % remove special status of '_' in ordinary text
\usepackage{parskip}      % handle non-indented paragraphs "properly"
\usepackage{array}        % new column definitions for tables
\usepackage[normalem]{ulem}
\usepackage{enumitem}
\usepackage{color}        % define colors for strikeouts and underlines
\usepackage{amsmath}      % additional math symbols
\usepackage{mathrsfs}     % mathscr font
\usepackage{microtype}
\usepackage{multicol}
\usepackage{xspace}
\usepackage{lmodern}
\usepackage[T1]{fontenc}
\usepackage[pdftex, final]{graphicx}
\usepackage[pdftex,
            pdftitle={C++ International Standard},
            pdfsubject={C++ International Standard},
            pdfcreator={Edward Smith-Rowland},
            bookmarks=true,
            bookmarksnumbered=true,
            pdfpagelabels=true,
            pdfpagemode=UseOutlines,
            pdfstartview=FitH,
            linktocpage=true,
            colorlinks=true,
            linkcolor=blue,
            plainpages=false
           ]{hyperref}
\usepackage{memhfixc}     % fix interactions between hyperref and memoir
\usepackage[active,header=false,handles=false,copydocumentclass=false,generate=std-gram.ext,extract-cmdline={gramSec},extract-env={bnftab,simplebnf,bnf,bnfkeywordtab}]{extract} % Grammar extraction

\pdfminorversion=5
\pdfcompresslevel=9
\pdfobjcompresslevel=2

\renewcommand\RSsmallest{5.5pt}  % smallest font size for relsize

%%--------------------------------------------------
%% fix interaction between hyperref and other
%% commands
\pdfstringdefDisableCommands{\def\smaller#1{#1}}
\pdfstringdefDisableCommands{\def\textbf#1{#1}}
\pdfstringdefDisableCommands{\def\raisebox#1{}}
\pdfstringdefDisableCommands{\def\hspace#1{}}

%%--------------------------------------------------
%% add special hyphenation rules
\hyphenation{tem-plate ex-am-ple in-put-it-er-a-tor name-space name-spaces non-zero}

%%--------------------------------------------------
%% turn off all ligatures inside \texttt
\DisableLigatures{encoding = T1, family = tt*}

\include{styles}

\title{Proposal to Introduce Two and Three Argument hypot Functions to \tcode{<complex>}}
%%\author{Edward Smith-Rowland}
\author{}
\date{}

\begin{document}
\chapterstyle{cppstd}
\pagestyle{cpppage}

\maketitle

\begin{verbatim}
Document Number: PnnnnR0
Date: 2017-03-06
Project: Programming Language C++
Audience: Library Working Group
Reply-to: Edward Smith-Rowland <3dw4rd@verizon.net>
\end{verbatim}

\section{Abstract}

This proposal seeks to introduce two-argument and three-argument hypoteneuse functions
to std::complex to match those functions for floating point types.

\section{Introduction and Motivation}

In complex mathematics and applications one often requires the distance between two points in the complex plane
for segments of contour integrals or the evaluation of complex functions.  In many fields of engineering and science
one requires the magnitude of three-dimensional complex vector fields which often vary over many orders of magnitude across
a region of space.  These distances and magnitudes are best computed in ways which prevent avoidable overflow and underflow.
Without these functions, programmers might be encouraged to take the square root of the sum of the squares and hope
for the best.

In addition, it is important to provide a uniform standard math library across all mathematical types in order to support
the implementation of type-generic mathematical, statistical, and engineering libraries.  To the greatest posible extent,
complex and real floating point types should share a suite of standalone functions even when some of them are no-ops
such as \tcode{std::real()} and \tcode{std::imag()} for foating point types.

Recall the distance formulas in two and three dimensions:

\[%
\mathsf{d}_2 = \sqrt{\lvert\mathsf{x}_2 - \mathsf{x}_1\rvert^2
                   + \lvert\mathsf{y}_2 - \mathsf{y}_1\rvert^2}
\]

\[%
\mathsf{d}_3 = \sqrt{\lvert\mathsf{x}_2 - \mathsf{x}_1\rvert^2
                   + \lvert\mathsf{y}_2 - \mathsf{y}_1\rvert^2
                   + \lvert\mathsf{z}_2 - \mathsf{z}_1\rvert^2}
\]

where the absolute value for complex numbers is used:
\[%
\lvert\mathsf{w}\rvert = \sqrt{Re[\mathsf{w}]^2 + Im[\mathsf{w}]^2}
\]
Note that $ \lvert\mathsf{w}\rvert $ is supported by \tcode{std::abs(w)}


\section{Design \& Considerations}

This proposal requires that the value types of all complex arguments be subjected to the usual
floating-point promotion rules.  This is the return type for the hypot functions.

One could further require that if any one variable is complex, any real variables are promoted to complex
and the promotion proceed as above.
I'm not proposing that now but it would be easy to add if it is desired.

This proposal does not add functions prefixed by \tcode{'c'} for complex or suffixed by type \tcode{'l'} or \tcode{'f'}.
The latter is in keeping with the real counterparts and this proposal seeks to widen genericity.


\section{Impact on the Standard}

This  proposal is a minimal library extension and does not require any changes to the core language,
nor will this affect code that depends on the current definition of \tcode{std::hypot} for floating-point types.
The complex variants of \tcode{std::hypot} is as mathematically well-understood as their real counterparts


\section{Proposed Wording}

Under
26.5.1	Header \tcode{<complex>} synopsis	[complex.syn]
...
// 26.5.7, values:
\begin{codeblock}
...
template<class T> T abs(const complex<T>\&);
template<class T> T arg(const complex<T>\&);
template<class T> T norm(const complex<T>\&);
+ template<class T> T hypot(const complex<T>\&, const complex<T>\&);
+ template<class T> T hypot(const complex<T>\&, const complex<T>\&, const complex<T>\&);
\end{codeblock}

Under
26.5.7	complex value operations	[complex.value.ops]

Insert

\begin{codeblock}
    template<class T> T hypot(const complex<T>\& x, const complex<T>\& y);
\end{codeblock}
/6	Returns: The two-dimensional hypoteneuse $\sqrt{x^2 + y^2}$.

\begin{codeblock}
    template<class T> T hypot(const complex<T>\& x, const complex<T>\& y, const complex<T>\& z);
\end{codeblock}
/7	Returns: The three-dimensional hypoteneuse $\sqrt{x^2 + y^2 + z^2}$.


\section{Feature Testing}

For the purposes of SG10, we recommend the feature-testing macro name \tcode{__cpp_lib_hypot}
added for real three-argument hypot be retained for this feature with a new date indicating
extension of both \tcode{std::hypot} overloads to \tcode{std::complex} types.


\section{Changes Since Revision 0 of the Proposal}

This is the first draft of this proposal.


\section{Acknowledgements}

I am grateful to Walter Brown for encouraging this proposal.


\section{References}

[N1836] Matt Austern, Draft Technical Report on C++ Library Extensions
http://www.open-std.org/jtc1/sc22/wg21/docs/papers/2005/n1836.pdf

[P0030R1] Benson Ma, Proposal to Introduce a 3-Argument Overload to std::hypot
http://www.open-std.org/jtc1/sc22/wg21/docs/papers/2015/p0030r1.pdf

\end{document}
